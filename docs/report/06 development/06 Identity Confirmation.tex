\section{Confirmation}

\subsection{Displaying Codes}

Instead of using the library in application code directly,
barcode functionality is abstracted behind an
\lstinline{IBarcodeService} to swap to different formats
and standards to replace QR if necessary.

Generation is handled using the QRCoder library, chosen for
its comprehensive range of configuration options and data
types for which it generates.
The barcode service uses the SVG format for two main
reasons: vector images render accurately across all devices
and displays, ensuring the confirmation process is reliable
and fast; and browsers have first-hand support for SVGs
which keeps their sizing and styling in the frontend.
See Figure \ref{fig:confirmationCode} for an example.

\begin{figure}[h]
  \centering
  \frame{
    \includegraphics[width=0.9\linewidth]{06
      development/assets/identity/confirmation page.png}}
  \caption{Confirmation page}
  \label{fig:confirmationCode}
\end{figure}

\subsection{Scanning Codes}

When a clock requires confirmation, the
\hyperref[s:providingLocation]{location process} is
followed as normal.
The scanning process is feature below as an additional
step, seen in Figure \ref{fig:scanCode}.

Also like the location process, it makes use of the Media
Streams API in the browser to access the camera on the
employee's device.
The jsQR library attempts to read a QR code in each frame
of the video and the clock can be submitted once scanned.

\begin{figure}[h]
  \centering
  \frame{\includegraphics[width=0.9\linewidth]{06
      development/assets/identity/clock page with camera
      feed.png}}
  \caption{QR scanning on the clock in/out page}
  \label{fig:scanCode}
\end{figure}