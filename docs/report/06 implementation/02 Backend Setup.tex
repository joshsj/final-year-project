\section{Backend Setup}

The architecture of the backend project is built around the
\gls{cqrs} pattern.
As described by \cite{cqrs}, it's ideally suited for a
\enquote{task-based UI} making it appropriate for
\projectname{} due to its focus on processes (e.g.,
clocking in, clocking out).

I chose Jason Taylor's \enquote{Clean Architecture}
template as a starting point, as I have previous
professional experience using it, it implements the
\gls{cqrs} pattern, and it is designed to serve an SPA
frontend using the latest .NET 6 technologies.
After removing some unnecessary features and replacing some
libraries for those with which I am more familiar, a
request to the API follows this structure:

\begin{enumerate} 

  \item A request is made: 

        {\small \lstinputlisting{06 implementation/assets/request
          process/request.txt} } 

  \item An endpoint is resolved, authenticated if required,
        and forwards the request to the CQRS implementation:

        {\small \lstinputlisting[language={[Sharp]C}]{06
          implementation/assets/request process/endpoint.cs} }

  \item The request is validated and authorized:

        {\small \lstinputlisting[language={[Sharp]C}] {06
          implementation/assets/request process/validation.cs} }

  \item A handler performs the actual functionality for the
        request: 

        {\small \lstinputlisting[language={[Sharp]C}] {06
          implementation/assets/request process/handler.cs}}
\end{enumerate} 

Each request is logged, additionally so if it causes
performance issues; exceptions are handled elegantly, and
exposed data is automatically mapped to DTOs to prevent
secure information being exposed unintentionally.

With the API project implemented, the
\enquote{\hyperref[ss:stories]{Backend Setup}} user story
is completed.
