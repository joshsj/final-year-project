\section{Project Setup}

\subsection{Backend}
The architecture of the backend project is built around the
\gls{cqrs} pattern.
As described by \cite{cqrs}, it's ideally suited for a
\enquote{task-based UI} making it appropriate for
\projectname{} due to its focus on processes (e.g.,
clocking in, clocking out).

I chose Jason Taylor's \enquote{Clean Architecture}
template as a starting point, as I have previous
professional experience using it, it implements the
\gls{cqrs} pattern, and it is designed to serve an SPA
frontend using the latest .NET 6 technologies.
After removing some unneeded features and swapping
Automapper for Mapster (familiarity), the API was set up.

Each request is logged, additionally so if it causes
performance issues; exceptions are handled elegantly, and
response data is automatically mapped to DTOs to prevent
secure information being exposed unintentionally.
The process for handling requests is as follows:

\begin{enumerate} 

  \item A request is made: 

        {\small \lstinputlisting{06 implementation/assets/request
          process/request.txt} } 

  \item An endpoint is resolved, authenticated if required,
        and forwards the request to the CQRS implementation:

        {\small \lstinputlisting[language={[Sharp]C}]{06
          implementation/assets/request process/endpoint.cs} }

  \item The request is validated and authorized:

        {\small \lstinputlisting[language={[Sharp]C}] {06
          implementation/assets/request process/validation.cs} }

  \item A handler performs the actual functionality for the
        request: 

        {\small \lstinputlisting[language={[Sharp]C}] {06
          implementation/assets/request process/handler.cs}}
\end{enumerate} 

With the API project implemented, the
\enquote{\hyperref[ss:stories]{Backend Setup}} user story
is completed.

\subsection{Frontend}

The project architecture in the frontend is more simple
than the backend, as the application itself is
comparatively simpler and Vue has a decided way of doing
things.

Navigation between pages is handled by Vue Router by
resolving routes in the same fashion as the API, also
guarding authenticated pages.
Global state managements is contained within a store
object, and complex functionality is  extracted from
components into plugins using the Composition API offered
in Vue 3.

\subsection{Communication}

Getting the two applications to talk to each other is made
much easier thanks to NSwag; a toolchain designed for the
.NET platform to generate OpenAPI schemas and HTTP clients
for C# and Typescript \parencite{nswag}.
As show in Figure \ref{fig:nswag}, every endpoint exposed
by the API has a corresponding class/method in Typescript
for use in the Vue app.

\begin{figure}[h]
  \centering

  \begin{subfigure}{\linewidth}
    \centering
    {\small\lstinputlisting[language={[Sharp]C}]{06
        implementation/assets/nswag/.cs}}
    \caption{API Endpoint}
  \end{subfigure}

  \begin{subfigure}{\linewidth}
    \centering
    {\small
      \lstinputlisting{06
        implementation/assets/nswag/.ts}}
    \caption{Generated Typescript client}
  \end{subfigure}

  \caption{NSwag Demonstration}
  \label{fig:nswag}
\end{figure}

The generated code is also extensible, which was required
to configure date parsing and the
\lstinline{authentication} header.

Another useful aspect of the .NET 6 ecosystem is first-hand
support to host \glspl{spa}.
Generally, a web app is hosted separately to the API it
serves which requires additional configuration for HTTPS
and CORS.
To negate these issues, .NET MVC can execute and
expose an \gls{spa} within the same server.
Microsoft documentation only explains the process for
Angular and React, but \cite{viteInDotnet} demonstrates
this functionality for any application using Vite.

With the frontend and backend project's set up, 
their \hyperref[ss:stories]{user stories} are done.

