\section{Implementation Process}

Discussions about new approaches in managing software
projects seem to surface every day, ranging from 
formal, traditional techniques like waterfall
through to the moving target of XP; landing in the middle
with user stories was a good choice.

Without a client for reference, describing functionality
from the user's perspective helped me envisage the solution,
and further specification from acceptance criteria and 
UI designs ultimately ensured I didn't miss any
functionality from the requirements. 

Organising said stories with Kanban also proved
useful, however a stricter approach to management would have
been better. Since I was developing the project alone, 
I neglected to estimate time/effort for the tasks as it all
had to get done at some point. However, working in sprints
with absolute deadlines would have helped my timekeeping
(as discussed in the \hyperref[s:pr]{personal reflection}.

Unsurprisingly, Git and Github proved to be invaluable; 
I couldn't imagine working without it. Integrating with Jira
was useful to find branches/pull requests at times and it's
worth the five minutes it takes to configure. Any longer and
I wouldn't miss it, however I can see it's value when
managing additional environments and deployments.

The stack of technologies used throughout the solution were
almost entirely familiar to me. Wherever new 
libraries/frameworks were required, their use of idiomatic
design patterns in C\#/Vue aided me in picking them up
easily and smoothly.