\section{Solution Structure}

The solution structure covers the main subsystems in the
application, any required external systems, and their
connections.

Determining the best structure for \projectname{} is a
balancing act.
The \hyperref[ss:goal]{project goal} sets a guideline to be
as universal as possible, yet the verification methodology
should be as strong as possible using the
\hyperref[s:concept]{technologies \& methodologies}
available from the research.
Hence, if a specific technology constrained the
universality of the application, should the goal or
methodology be prioritised?

\subsection{Architecture}

\paragraph{Backend}
As discussed by \cite{soaVsMicroservices}, trends in modern
software development are leaning towards microservices and
server-oriented architecture (SOA), with the main
advantages of separation and scalability.

Given the small scope of the \projectname{} application, a
simpler and easier approach is more suitable, so a monolith
will form the backend of the \projectname{} system; i.e., a
single container for most/all functionality, decomposed
into separate components internally \parencite{monoliths}.

The functionality contained within the backend will be
presented using an API following REST principles, primarily
because I have the most experience with REST by far.
It's also \enquote{simple, well-known, and widely used}
\parencite{protocols}, and it's rivals are legacy (SOAP);
opinionated and restrictive in approach (GraphQL); and only
supported on HTTP/2 (gRPC).

Alongside the server application, a database will also be
required for data persistence and file system access for
handling configuration files.

\paragraph{Frontend}
The device coverage and compatibility provided by the web
is unparalleled, so a web application is the ideal
solution.
Modern browsers support access to device hardware such as
the camera \parencite{browserCamera}, enabling all the
remaining researched technologies, and interacting with an
API is standard practice on the web.

A caveat of developing website is, at present, no support
for NFC; an alternative could be a native mobile
application for Android and/or iOS, as the both provide
APIs for using NFC.
On balance, limiting \projectname{}'s compatibility to
these operating systems does not outweigh the value of NFC
support, as barcodes provide a worthy alternative.

\subsection{Technologies}

The variety in languages \& frameworks for developing
contemporary web applications is huge: any can accomplish
the project goal.
Ergo, the most important factor for the following decisions
is my familiarity and comfortability with the framework and
its language(s).

\paragraph{Backend}

Frameworks for creating web APIs are seemingly endless:
Django, Flask, Ruby on Rails, Laravel, etc. I have the most
experience using ASP.NET (natively) and Node.js with
express.

Both platforms have a wealth of community support,
including project templates, used to architect solutions
cleanly and effectively; and libraries aimed at validation,
authentication, authorization, databases plus ORM/ODM,
CORS, etc. They are also both cross-platform, meaning they
can be deployed on Windows or UNIX subsystems.

One difference between the platforms is performance: across
a range of tests, \cite{webFrameworkBenchmarks} allocated
relative scores of 60.7\% to .NET Core and 15.6\% to
Node.js.

Another difference is their language.
C\# is the main choice for .NET Core, and Javascript, plus
Typescript, for Node.js.
Both languages share some features (e.g., types plus
generics, async/await, lambdas, functional array methods,
null handling operators) \parencite{csharp,ts}, however C\#
preserves its types at runtime, which is inherently safer
and enables reflection.
Typescript offers more functional approaches natively, with
null typing and unions, but these features are appearing in
C\# also.

Overall, my preference lies with .NET Core; libraries like
Entity Framework, LINQ, and Identity Server handle an APIs
most complex functionality better than the Node.js
alternatives, and it performance benefits should result in
a smoother experience using \projectname{} as a whole.

\paragraph{Frontend}
Websites serving an API can be constructed using \gls{ssr}
or a \gls{spa}; see Table \ref{tbl:webTechComparison} for a
comparison between their functionalities.

\begin{table}[h]
  \centering
  \small
  \begin{tabular}{lll}
  & \gls{spa} & \gls{ssr} \\ \hline
  
  \makecell[l]{Complex page\\functionality} & \makecell[l]{Endless support with\\frameworks \& libraries} & Limited to native HTML \& Javascript \\

  Load times & \makecell[l]{Initial is slow,\\subsequent are fast} & Average \\

  SEO & Bad: all content is dynamic & Good: most content can be cached \\

  \makecell[l]{Browser\\agnostic} & \makecell[l]{No: affected by the browser's\\Javascript engine} & Yes: only HTML is served \\
\end{tabular}

  \caption{Frontend Web Technology Comparison}
  \label{tbl:webTechComparison}
\end{table}

Personally, I have decent experience with SSR using
WebForms in ASP.NET and moustache
templates within PHP and Javascript
frameworks, but my experience with Vue.js
surpasses both, as well as other SPAs like React and
Angular.
An SPA is also more suited to the discreet functionality of
the verification methodology, so the frontend will be
developed with Vue.

% architecture overview
