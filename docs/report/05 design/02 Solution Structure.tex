\section{Solution Structure}

The solution structure covers the main subsystems in the
application, any required external systems, and their
connections.

Determining the best structure for \projectname{} is a
balancing act.
The \hyperref[ss:goal]{project goal} sets a guideline to be
as universal as possible, yet the verification methodology
should be as strong as possible using the
\hyperref[s:concept]{technologies \& methodologies}
available from the research.
Hence, if a specific technology constrained the
universality of the application, should the goal or
methodology be prioritised?

\subsection{Architecture}

\paragraph{Backend}
As discussed by \cite{soaVsMicroservices}, trends in modern
software development are leaning towards microservices and
server-oriented architecture (SOA), with the main
advantages of separation and scalability.

Given the small scope of the \projectname{} application, a
simpler and easier approach is more suitable, so a monolith
will form the backend of the \projectname{} system; i.e., a
single container for most/all functionality, decomposed
into separate components internally \parencite{monoliths}.

The functionality contained within the backend will be
presented using an API following REST principles, primarily
because I have the most experience with REST by far.
It's also \enquote{simple, well-known, and widely used}
\parencite{protocols}, and it's rivals are legacy (SOAP);
opinionated and restrictive in approach (GraphQL); and only
supported on HTTP/2 (gRPC).

Alongside the server application, a database will also be
required for data persistence and file system access for
handling configuration files.

\paragraph{Frontend}
The device coverage and compatibility provided by the web
is unparalleled, so a web application is the ideal
solution.
Modern browsers support access to device hardware such as
the camera \parencite{browserCamera}, enabling all the
remaining researched technologies, and interacting with an
API is standard practice on the web.

A caveat of developing website is, at present, no support
for NFC; an alternative could be a native mobile
application for Android and/or iOS, as the both provide
APIs for using NFC.
On balance, limiting \projectname{}'s compatibility to
these operating systems does not outweigh the value of NFC
support, as barcodes provide a worthy alternative.

\subsection{Technologies}

The variety in languages \& frameworks for developing
contemporary web applications is huge: any can accomplish
the project goal.
Ergo, the most important factor for the following decisions
is my familiarity and comfortability with the framework and
its language(s).

\paragraph{Backend}

Frameworks for creating web APIs are seemingly endless:
Django, Flask, Ruby on Rails, Laravel, etc. I have the most
experience using ASP.NET (natively) and Node.js with
express.

Both platforms have a wealth of community support,
including project templates, used to architect solutions
cleanly and effectively; and libraries aimed at validation,
authentication, authorization, databases plus ORM/ODM,
CORS, etc. They are also both cross-platform, meaning they
can be deployed on Windows or UNIX subsystems.

One difference between the platforms is performance: across
a range of tests, \cite{webFrameworkBenchmarks} allocated
relative scores of 60.7\% to .NET Core and 15.6\% to
Node.js.

Another difference is their language.
C\# is the main choice for .NET Core, and Javascript, plus
Typescript, for Node.js.
Both languages share some features (e.g., types plus
generics, async/await, lambdas, functional array methods,
null handling operators) \parencite{csharp,ts}, however C\#
preserves its types at runtime, which is inherently safer
and enables reflection.
Typescript offers more functional approaches natively, with
null typing and unions, but these features are appearing in
C\# also.

Overall, my preference lies with .NET; libraries like
Entity Framework, LINQ, and Identity Server handle an APIs
most complex functionality better than the Node.js
alternatives, and it performance benefits should result in
a smoother experience using \projectname{} as a whole.

\paragraph{Database}
For the \projectname{} application, an \gls{rdb} using SQL
is the better choice, as opposed to a \gls{nrdb} (No-SQL)
offering.
\cite{databaseComparison} identifies scalability, dynamic
models, and data abstraction as pros of No-SQL, whereas SQL

has a smaller data footprint and strong integrity.
The advantages of No-SQL are almost irrelevant to a small
system like \projectname{}, and data integrity with a
reduced footprint is never a bad thing.

% TODO add testing frameworks ?

\paragraph{Frontend}
Websites serving an API can be constructed using \gls{ssr}
or a \gls{spa}; see Table \ref{tbl:webTechComparison} for a
comparison between their functionalities.

\begin{table}[h]
  \centering
  \small
  \begin{tabular}{lll}
  & \gls{spa} & \gls{ssr} \\ \hline
  
  \makecell[l]{Complex page\\functionality} & \makecell[l]{Endless support with\\frameworks \& libraries} & Limited to native HTML \& Javascript \\

  Load times & \makecell[l]{Initial is slow,\\subsequent are fast} & Average \\

  SEO & Bad: all content is dynamic & Good: most content can be cached \\

  \makecell[l]{Browser\\agnostic} & \makecell[l]{No: affected by the browser's\\Javascript engine} & Yes: only HTML is served \\
\end{tabular}

  \caption{Frontend Web Technology Comparison}
  \label{tbl:webTechComparison}
\end{table}

Personally, I have decent experience with SSR using
WebForms in ASP.NET and moustache templates within PHP and
Javascript frameworks, but my experience with Vue.js
surpasses both, as well as other SPAs like React and
Angular.
An SPA is also more suited to the discreet functionality of
the verification methodology, so the frontend will be
developed with Vue.

% TODO add testing frameworks ?

% TODO architecture overview

\subsection{Management}

\cite{agileVsTraditional} describe five strategies for
project management in software applications: two are
traditional, three are agile.

\paragraph{Traditional}

The linear project management strategy is a traditional
approach consisting of \enquote{dependent, sequential
  phases that are executed with no feedback loops}; an
infamous example is the waterfall method which strictly
follows a phase of: (1) requirements, (2) analysis, (3)
design, (4) development, (5) testing, and (6) operations.
There is a single solution, produced at the end of the
final phase.

A similar approach is the incremental strategy, which
follows the same processes but releases a partial solution
at the end of each phase, producing business value much
earlier to accommodate for change requests.

Ultimately, traditional strategies are long, rigid
processes which rely on a \enquote{clearly defined goal,
  solution, and requirements}.

\paragraph{Agile}
In contrast, agile approaches allow for ignorance in the
project goal and/or requirements and/or end-solution.

With an iterative strategy, the goal and requirements are
known upfront.
To guide the end-solution towards the requirements, a
feedback loop with the customer is introduced following
groups of phases to receive suggestions and accommodate for
scope changes.

The adaptive strategy takes this further.
Both the end-solution and requirements are unknown, so the
feedback loop determines the requirements for the
subsequent phases, knows as \enquote{continuous change}.
Scrum is considered an adaptive strategy.

Finally, extreme strategies know nothing upfront; the range
of acceptable solutions is wide; feedback loops redefine
the scope of the project as a whole.

On balance, an iterative strategy is most appropriate to
create \projectname{}; the \hyperref[ss:goal]{goal} and
\hyperref[s:requirements]{requirements} are known, but the
end-solution is subject to change throughout development
and testing.

\paragraph{Decomposition}

For development and testing, the workload will first be
broken down into epics; i.e., a major piece of
functionality within the project such as
\enquote{Authentication} \parencite{agile}.
Epics only require a description for additional
information/context.

Within an epic, the work will be decomposed into user
stories: a feature described from a user's perspective
\parencite{agile}.
They will include: a description, also for additional
information/context; any wireframes for the frontend UI;
and acceptance tests, which state how the system should
function against which to test.

Within a story, tasks will represent the actual work.
A task is a small, achievable unit of work.

\paragraph{Tracking}

In alignment with agile, Kanban boards allow decomposed
work follow a set workflow to track progress.
They are structured with columns which represent status
(e.g., \enquote{Todo}, \enquote{In Progress} \&
\enquote{Done}), with task displayed as a card
\enquote{card} to show its most important information
\parencite{kanban}.
See Table \ref{tbl:kanbanSprint} for a comparison to Scrum.

\begin{table}[h]
  \centering
  \begin{tabular}{lll}
  & Kanban & Sprint \\ \hline
  
  Time Period & Ongoing & Defined start and end \\
  
  Team roles & None & \makecell[l]{Product owner,\\Development team,\\Scrum master} \\
  
  \makecell[l]{Task\\designation} & \makecell[l]{Created, modified,\\assigned as needed} & Established at start of sprint  \\
  
  Task limit & No & Yes (strict)
  
\end{tabular}
  \caption{Kanban vs. Scrum}
  \parencite{kanban}
  \label{tbl:kanbanSprint}
\end{table}
