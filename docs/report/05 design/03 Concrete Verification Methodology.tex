\section{Concrete Verification Methodology}

\subsection{Identity}

Identifying users will rely on
\hyperref[s:userAuth]{existing user authentication
  methodologies}.

The first option is a custom implementation; broadly
speaking, the development community (rightly) discredits
this approach.
\cite{noCustomAuth} identifies these drawbacks:

\begin{itemize}

  \item Complexity --- implementing different credentials,
        integrated into multiple
        factors, in accordance to industry-wide
        protocols/standards is difficult

  \item Risk --- data breaches will discourage new users
        and tarnish customer trust; incompatibility with
        industry standards can deter enterprise clients

\end{itemize}

Alternatively, the .NET 6 ecosystem has a comprehensive
identity platform, encompassing \enquote{users, passwords,
  profile data, roles, claims, tokens, email confirmation},
plus project templates for \gls{spa} technologies
\parencite{netIdentity}.
This is a viable solution, but it too has drawbacks: no SPA
template for Vue exists, and altering a template would be
difficult and lengthy; the components \& pages for
authentication are implemented in Razor pages alongside the
SPA, which are difficult to debug and customise.

The Auth0 \gls{identityProvider} will instead be
responsible for user authentication, outsourcing a lot of
application complexity and improving usability.
By default, Auth0 provides its own account registration \&
sign-in mechanism, but also offers external providers
(e.g., Google, LinkedIn, Twitter).
Once signed in, user data is stored on their servers,
accessed via their API.
Like .NET, an authentication process (or \enquote{flow})
for an Auth0 application can be configured for SPAs, and
they maintain Javascript packages to integrate these flows.

% TODO check these are the right accounts

\paragraph{Associating Users}
Some associations in a \gls{rdb} require the data to exist.
This presents an problem to associate users, as Auth0
stores user data separate from the application's database.

To enable such associations, the \code{User} table will
only store basic user information (e.g., name) and the
unique identifier for their user on Auth0.
This data is updated within the
\hyperref[p:authFlow]{authentication flow}, as shown in
Figure \ref{fig:auth0Data}.

\begin{figure}
  \lstinputlisting{05 design/assets/auth0 jwt.json}
  \caption{Auth0 User Data}
  \captiondesc{Specifically, a JWT for a new account
    created with a Github no-reply email; the \code{sub}
    field is the identifier for the user.}
  \label{fig:auth0Data}
\end{figure}

\paragraph{Using Social Accounts}
To ensure valid user data is present, the normal login
mechanism using a dedicated Auth0 account will be disabled,
replaced with Google and Apple login.
Given the popularity of device requiring these accounts
(i.e., any Apple device, Gmail, Android phones), it's
highly unlikely for the average user not to have at least
one available account.

\paragraph{Authentication Flow}
\label{p:authFlow}

\begin{enumerate}

  \item An employee opens the \projectname{} website and
        clicks \enquote{register} or \enquote{login}

  \item The page redirects to Auth0, offering social
        services with which to authenticate

  \item Once signed in, they are redirected back to the
        \projectname{} site

  \item In the background:
        \begin{enumerate}
          \item An API call from the frontend notifies the
                backend to update the employee's
                personal data

          \item The API retrieves the employee's
                information from Auth0 and persists it
        \end{enumerate}

  \item The employee can now use the website
\end{enumerate}

% TODO flowchart

\subsection{Location}

At the core of

% Job Location:
% stored at lat, long, radius
% configurable accuracy threshold, based on research, configurable, tested 
% employee assigned to them

% gps

\subsection{Confirmation}

% factors