\section{Solution Structure}

The solution structure establishes the main subsystems in
the application, any required external systems, and their
connections.

Determining the best structure for \projectname{} is a
balancing act.
The \hyperref[ss:goal]{project goal} sets a guideline to be
as universal as possible, yet the verification methodology
should be as strong as possible using the
\hyperref[s:concept]{technologies \& methodologies}
available from the research.
Hence, if a specific technology constrained the
universality of the application, should the goal or
methodology be prioritised?

Fortunately, the only technology which presents this
challenge is \hyperref[ss:nfc]{NFC}, due to its reduced
availability in devices.
Since it was concluded as an alternative to barcodes, its
benefits do not outweigh its restriction on
\projectname{}'s device compatibility.

\subsection{Backend}

As discussed by \cite{soaVsMicroservices}, trends in modern
software development are leaning towards microservices and
server-oriented architecture (SOA), with the main
advantages of separation and scalability.

Given the small scope of the \projectname{} application, a
simpler and easier approach is more suitable, so a monolith
will form the backend of the \projectname{} system; i.e., a
single container for most/all functionality, decomposed
into separate components internally \parencite{monoliths}.

The functionality contained within the backend will be
presented using an API following REST principles, primarily
because I have the most experience with REST by far.
It's also \enquote{simple, well-known, and widely used}
\parencite{protocols}, and it's rivals are legacy (SOAP);
opinionated and restrictive in approach (GraphQL); and only
supported on HTTP/2 (gRPC).

Alongside the server application, a database will also be
required for data persistence.

\subsection{Frontend}

The device coverage and compatibility provided by the web
is unparalleled, so a web application is the ideal
solution.
Modern browsers support access to device hardware such as
the camera \parencite{browserCamera}, enabling all the
remaining researched technologies, and interacting with an
API is standard practice on the web.

% architecture overview
