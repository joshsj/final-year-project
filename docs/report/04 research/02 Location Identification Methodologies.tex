\section{Location Identification Technologies}

Location identification simply refers to establishing the
physical location of a device on the planet.
In theory, this also includes the location of the user
associated with the device.

\subsection{Cellular Positioning}
\label{ss:cellPos}
Cellular positioning uses the Global System for Mobile
communication (or GSM): \enquote{a digital mobile network
  that is widely used by mobile phone users in Europe and
  other parts of the world} \parencite{whatIsGSM}.

Devices using the GSM network require a SIM card,
conventionally found in mobile phones.
The SIM card is identified using its International Mobile
Subscriber Identity (IMSI), and the phone with its
International Mobile Equipment Identity (IMEI)
\parencite{trackingSuspectByPhone}.

The GSM network consists of base stations with fixed
geographical positions, or 'cell towers'.
The distance of a connected device from a cell tower can be
approximated using Time of Arrival (ToA)\label{t:toa} and
Angle of Arrival (AoA); ToA measures the time taken for a
signal to reach a device, and AoA measures the angle of
received signals to compare against the geography within
range of the tower \parencite{suveryOfCellPos}.

More accurate measurements use cell tower
\gls{triangulation} \parencite{howCellTowerTriWorks}, or
\gls{trilateration} \parencite{suveryOfCellPos}).
When a device is within range of three cell towers, the
overlap in the towers' ranges can establish its location.
According to \cite{locationComparison}, cellular
positioning was the least accurate technology to find an
iPhone 3G's location with a \enquote{median error of 600 m
  for 64 observations}.

its location up to an accuracy of \nicefrac{3}{4} miles
\parencite{howCellTowerTriWorks}, though less accurate
measurements can be taken with only two.

Device's themselves and the GSM infrastructure can
determine a device's location, hence its use by police
forces in America \parencite{howCellTowerTriWorks}, however
access to the infrastructure is restricted to paying
customers (e.g., mobile cell carriers) and law enforcement.

\subsection{GPS}

Similar to \hyperref[ss:cellPos]{cellular positioning}, the

Global Positioning System (GPS) uses 24 satellites in
earth's orbit to locate devices via \gls{trilateration}.
Unlike cell positioning, it calculates distance exclusively
using the \hyperref[p:toa]{time taken} by a signal
\parencite{suveryOfCellPos})}.

\cite{locationComparison} states \enquote{most newer model
  cell phones are GPS-enabled}; given the age of the report
and the ever-increasing capabilities of smartphones, GPS
has become an everyday technology.
To reduce power consumption and improve
'time-to-first-fix', most smartphones implement Assisted
GPS (A-GPS), which delegates many complexities of a GPS
receiver to the server with a sacrifice to accuracy.

Although \cite{locationComparison} determined A-GPS to be
less accurate than a standard GPS receiver, with an
\enquote{average median error of 8 m}, it concluded A-GPS
appears sufficient for most location-based services.

\subsection{Geofencing}

\subsection{Bluetooth Beacons}

\subsection{IP Address-based}