\section{Location Identification Technologies}

Location identification simply refers to establishing the
physical location of a device on the planet.
In theory, this also includes the location of the user
associated with the device.

\subsection{Cellular Positioning}

Cellular positioning uses the Global System for Mobile
communication (or GSM): \enquote{a digital mobile network
  that is widely used by mobile phone users in Europe and
  other parts of the world} \parencite{whatIsGSM}.

Devices using the GSM network require a SIM card, typically
found in mobile phones.
The SIM card is identified using its International Mobile
Subscriber Identity (IMSI), and the phone with its
International Mobile Equipment Identity (IMEI)
\parencite{trackingSuspectByPhone}.

The GSM network consists of base stations with fixed
geographical positions.
Each tower is allocated a Cell-ID and a Location Area
Identifier (LAI) to uniquely identify a tower and its
geographic position respectively.
This is the simplest way to identify a device's physical
location using GSM \parencite{suveryOfCellPos}.

More accurate measurements use cell tower
\enquote{triangulation} \parencite{howCellTowerTriWorks},
or \enquote{trilateration} \parencite{suveryOfCellPos}).
When a device is within range of three cell towers, the
overlap of its distances from each can establish its
location up to an accuracy of \nicefrac{3}{4} miles
\parencite{howCellTowerTriWorks}.

\subsection{GPS}

\subsection{Geofencing}

\subsection{Bluetooth Beacons}

\subsection{IP Address-based}

\subsection{Conclusions}

% cant trust the cient
