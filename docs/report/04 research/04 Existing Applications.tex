\section{Existing Applications} \label{s:existing}

The following applications are \gls{time-clock} systems,
all of which employ security mechanisms to prevent
\glsdisp{buddy-punch}{buddy punching}.

\subsection{Dedicated Time Clock Devices}

Clocking into one's shift at work has been around since
1888, after Willard Legrand Bundy invented the
\gls{time-clock}.
Since then, time clocks have been developed to integrate
with spreadsheets and payroll systems
\parencite{clockingHistory}.

Looking at time clocks from ClockingSystems
\parencite{clockingSystems} and Time Systems UK
\parencite{timeSystemsUk}, modern devices employ complex
security mechanisms using many aforementioned technologies:
\hyperref[ss:biometrics]{biometrics}, using fingerprints,
hand readers, palm readers, and face recognition; and
proximity-based solutions using \hyperref[ss:rfid]{RFID}.
Devices can communicate using WiFi and GSM.
They offer devices for permanent installation (i.e., on a
wall) or robust portable devices to use on-location.

They also feature business functionalities such as
auto-deductions for lateness, auto-deductions for lunch
breaks, handling of sickness, and reporting, plus
integrations with other business applications such as Sage
and Jira.

\subsection{Time Clock Applications}

Time clocks are also available to companies as online
applications, offering the same business functionalities
and integrations as the device-based options above.
They can be used on the web, or using native applications
for iOS and Android.

Among the many example of these application (e.g.,
QuickBooks Time, Connecteam, Hubstaff, Clockify),
BuddyPunch is the most transparent about the security of
their system through their documentation
\parencite{buddyPunchDocs}.

Their main security component uses \hyperref[ss:gps]{GPS}
to determine employee locations, optionally with
\hyperref[ss:geofencing]{geofences} to define job
locations.
In their FAQ about spoofing GPS data, they only specify
their Android safeguard; \enquote{[users] must disable Mock
  Locations first in order to punch time}.

Another security component uses pictures taken of them whe
clocking in/out, with an option to configure physical
\enquote{positions} to ensure the photo is taken
at-present.

The device lock feature allows managers to \enquote{lock
  employees to specific devices}.
It is only available through the browser, as it stores a
unique cookie to identify the device.
Similarly, an IP address lock is also offered.

Lastly, facial recognition is offered using their
proprietary system or FaceID on iOS devices.

\subsection{Conclusions}

Dedicated devices is simply irrelevant to \projectname{};
they are, by nature, an implementation for a single company
which opposes the \enquote{universal} guideline stated in
the \hyperref[ss:goal]{project goal}.

Unsurprisingly, existing time clock applications share some
overlap with this project identically; they clock-in and
clock-out employees.
However, the BuddyPunch system is sold as a service to
individual companies, as opposed to the goal of a shared
system among all companies using \projectname{} meaning the
systems must be much less susceptible to false clock-ins.

Facial recognition is a reliable mechanism for identifying
people, but it inherently requires very personal data to be
stored by a third party who is objectionable. It also has
huge safety implications if leaked so it's protection is
crucial. Lastly, its reliability and accuracy depends on:
camera availability and quality, lighting conditions, and
facial access (i.e., absence of masks, dirt, religious
covering, disfigurement). All in all, facial recognition
applies too many constraints to use in this project.

GPS is ultimately circumventable, regardless of safeguarding
as it's a client-side technology. Device locking is complete
unfeasible for \projectname{} as it is aimed to be
device/infrastructure independent, plus copying a cookie or
spoofing an IP is also achievable with minimal expertise. 

% TODO sources