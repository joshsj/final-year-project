\section{Location Identification Technologies}

Location identification simply refers to establishing the
physical location of a device on the planet.
In theory, this also includes the location of the user
associated with the device.

All the technologies explored below follow the same basic
principle: 

\begin{itemize} 

  \item An infrastructure is composed of signal
        transmitters/receivers			

  \item Devices in-range of transmitter(s) send/receive data
        using signals 

  \item The infrastructure and/or devices analyse the signal
        to approximate the device's location 

\end{itemize} 

\subsection{Cellular Positioning} \label{ss:cellPos}
Cellular positioning uses the Global System for Mobile
communication (or GSM): \enquote{a digital mobile network
  that is widely used by mobile phone users in Europe and
  other parts of the world} \parencite{whatIsGSM}.
% %%

Devices using the GSM network require a SIM card,
conventionally found in mobile phones.
The SIM card is identified using its International Mobile
Subscriber Identity (IMSI), and the phone with its
International Mobile Equipment Identity (IMEI)
\parencite{trackingSuspectByPhone}.

The infrastructure of the GSM network consists of base
stations with fixed geographical positions on the planet's
surface, known as cell towers; \gls{aoa}, \gls{toa}, and
\gls{tdoa} can be used to calculate the geographical
location of a device by comparing against the geography
within range of the tower.
The accuracy of these calculations is improved using
\gls{triangulation} \parencite{howCellTowerTriWorks}, or
\gls{trilateration} \parencite{suveryOfCellPos}).

According to \cite{locationComparison}, cellular
positioning was the least accurate technology to find an
iPhone 3G's location with a \enquote{median error of 600 m
  for 64 observations}.

Only the GSM infrastructure can determine a device's
location, however access to the infrastructure is
restricted to paying customers (e.g., mobile cell carriers)
and law enforcement \parencite{howCellTowerTriWorks}.

\subsection{GPS} \label{ss:gps}

Similar to \labelref{ss:cellPos}{cellular positioning},
Global Positioning System (GPS) uses 24 satellites in
earth's orbit to locate devices via \gls{trilateration}.
Unlike cell positioning, it calculates distance exclusively
using \gls{toa} \parencite{suveryOfCellPos}.

\cite{locationComparison} states \enquote{most newer model
  cell phones are GPS-enabled}; given the age of the report
and the ever-increasing capabilities of smartphones, GPS
has become an everyday technology.
To reduce power consumption and improve
\enquote{time-to-first-fix}, most smartphones implement
Assisted GPS (A-GPS), which delegates many complexities of
a GPS receiver to the server with a sacrifice to accuracy.

Although \cite{locationComparison} determined A-GPS to be
less accurate than a standard GPS receiver, with an
\enquote{average median error of 8 m}, it concluded A-GPS
appears sufficient for most location-based services.

\subsection{Geofencing}

According to \cite{whatIsGeofencing}, a geofence is a
\enquote{virtual geographic boundary around a physical
  location}.
Geofencing is an aspect of \gls{contextualAwareness} in
mobile devices, enabling functionality to trigger depending
on a user's location.
For example, customers expecting a package can be alerted
when its courier is approaching.

Any device with a GPS connection can use geofencing
functionality; both Android and iOS have native
implementations to detect/embed geofences into apps
\parencite{androidGeofencingApi,iosGeofencingApi}.
Inherently, the accuracy of geofencing is defined by the
accuracy of its GPS receiver.

\subsection{WiFi Positioning}

As opposed to maintaining a global infrastructure like
\labelref{ss:gps}{GPS}, WiFi positioning uses WLAN access
points (AP) instead, which has a major advantage of indoor
coverage in comparison to GPS and
\labelref{ss:cellPos}{cellular positioning}
\parencite{locationComparison}.

WiFi positioning can use \gls{trilateration} via \gls{aoa}
and \gls{toa} to determine locations, as well as
fingerprinting.
Since APs do not store their own geographical position,
external services/resources are required to acquire this
data.
Instead, fingerprinting places a device at a known
location, and gathers the signal strengths to nearby APs
using \gls{rssi} (the calibration phase).
This dataset allows a device to determine its geographical
position at unknown location by comparing its current
signal strengths to the dataset (the positioning phase)
\parencite{locationComparison}.

Across 58 observations, \cite{locationComparison} found a
median error of 74m using WiFi positioning, also revealing
\enquote{erratic spatial patterns resulting from the design
  of the calibration effort underlying the WiFi positioning
  system}.

\subsection{Bluetooth Beacons}

Beacons are a small, one-way, signal transmitter.
A bluetooth beacon specifically uses \gls{ble} to transmit
small packets of data at regular intervals.
This data is only usable by software designed for its
application; e.g., an app for a museum which provides
information about the nearest exhibits
\parencite{usingBluetoothBeacons}.

\cite{bluetoothBeaconAccuracy} investigated the accuracy of
5 commercial bluetooth beacons, using 23 smartphones at
ranges of 1, 2, and 10 meters, concluding an average
accuracy \enquote{between 0.79 meters to 2.28 meters}.